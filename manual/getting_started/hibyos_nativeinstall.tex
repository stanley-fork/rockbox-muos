In order to install Rockbox onto the \dap{}, we need to first install the bootloader
onto the device's internal memory. The bootloader is the only modification to the device
we will make - Rockbox itself will exist only on the SD card.

Once the bootloader is installed onto the device, it is exceedingly rare to need to change it.
Updating Rockbox is as simple as overwriting it on the SD card and rebooting the device.

We will install the bootloader with the original firmware's recovery
mode. Please follow the instructions detailed below.
% mode. The process is summed up by the following:

% \begin{enumerate}
%   \item Determine what hardware version your player is and download
%     the correct bootloader update file
%   \item Place the bootloader \fname{update.upt} file on the SD card
%   \item In the original firmware, run the Firmware Update: \\
%     \fname{System Settings --> Firmware Update}
% \end{enumerate}

% \textbf{These steps, in detail, are:}

\begin{enumerate}
\item \textbf{Determine hardware version}

Determine what hardware version your player is. Go to \fname{System Settings --> About The Player --> Version} and reference
the list below. hw1, hw1.5, and hw2 players all use the same update
file (with one exception), while hw3 players and hw4 players each use a different one.

\note{Important: If your player's version is not contained in this list,
for example if the firmware version is newer than what is listed here,
we cannot be sure that the hardware is the same. The best thing to do is
contact the manufacturer and ask them two things: (1) for an update file
of your version, and (2) if a player with the most recent version listed
here can be upgraded to the firmware version on your player. If they say
yes, we can be more certain that the hardware has not changed. These lists
may not be the most up to date, please see the wiki page at
\url{https://www.rockbox.org/wiki/AIGOErosQK} for the most up-to-date list.}

\begin{description}
\item[hw1/hw1.5/hw2 players]
\hfill{}
  \begin{itemize}
    \item Aigo Eros Q V1.8 - V2.0
    \item Hifiwalker H2 V1.1 - V1.6
    \item Surfans F20 V2.2 - V2.7
  \end{itemize}
  These players use \fname{erosqnative-hw1hw2-erosq.upt} as the update file.
  The lone exception is the Hifiwalker H2 V1.3, which uses the update file \\
  \fname{erosqnative-hw1hw2-eros\_h2.upt}.
\item[hw3 players]
\hfill{}
  \begin{itemize}
    \item Aigo Eros Q V2.1
    \item Hifiwalker H2 V1.7 - V1.8
    \item Surfans F20 V3.0 - V3.3
  \end{itemize}
  These players use \fname{erosqnative-hw3-erosq.upt} as the update file.
\item[hw4 players]
\hfill{}
  \begin{itemize}
    \item Aigo Eros Q V2.2
    \item Hifiwalker H2 V1.9 - V2.0
    \item Surfans F20 V3.4
  \end{itemize}
  These players use \fname{erosqnative-hw4-erosq\_2024.upt} as the update file.
\end{description}

Download the \fname{.upt} file for these players from \download{bootloader/aigo/native/}.

% \subsubsection{Format the SD card with the stock firmware}\label{ref:format_sd_card}
% This is a convenient time to format the SD card. We need to ensure the SD card is formatted
% as FAT - most cards come from the factory with exFAT on them. The stock firmware can be used to do this.
% Back up any file you do not want to lose and go to \fname{System Settings --> Reset --> Format TF Card}
% in the stock firmware. Once done, you can put your files back on the SD card.

% \note{Remember, this must also be done if you acquire a new SD card in the future!}

\item \textbf{Place update file on SD card}

Place the appropriate bootloader file on the root of the SD card and name it
\emph{exactly} \fname{update.upt}.

Don't forget to safely eject/unmount your player.

\item \textbf{Run Firmware Update}

In the original firmware, run the firmware updater by going to
\fname{System Settings --> Firmware Update}. The player will reboot into the updater and self-install
the bootloader.

If you have not installed the Rockbox build onto the SD card yet, it will reboot and fail
to find Rockbox. We simply then need to install Rockbox onto the SD card. Please continue to
section \ref{sec:installing_firmware}, \textit{Installing the Firmware}.

Once the bootloader is installed onto the device, \fname{update.upt} can be deleted from the SD card if you wish.
\end{enumerate}
